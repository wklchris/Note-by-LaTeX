% safe参数解决与\!在内的多个冲突
% \sups命令可能被重定义,xeCJK放在tipa后
\RequirePackage[safe]{tipa}
\documentclass[a4paper, zihao=-4, linespread=1]{ctexrep}
  \xeCJKsetup{CJKmath=true}
  \renewcommand{\CTEXthechapter}{\thechapter}


% 中文支持
% \iffalse与\fi可以实现“多行注释”
\iffalse
\usepackage[slantfont,boldfont]{xeCJK}
	\setCJKmainfont[BoldFont=SimHei,ItalicFont=KaiTi]{SimSun}
	\setCJKmathfont{STXinwei}
\usepackage{indentfirst}
\fi
\setCJKmathfont{KaiTi}
\newCJKfontfamily[zhxinwei]\xinwei{STXinwei}

% 数学环境
\usepackage{amsmath}
  \newcommand{\ue}{\mathrm{e}}
  \newcommand{\ud}{\mathop{}\negthinspace\mathrm{d}}
\usepackage{amssymb}
\usepackage{mathrsfs} % 线性代数字体
    % overline的替代命令
\newcommand{\closure}[2][3]{{}\mkern#1mu\overline{\mkern-#1mu#2}}
\usepackage{yhmath} % 左下-右上省略号
\usepackage{mathtools} % dcases环境
\usepackage{amsthm} % 定理环境
  \theoremstyle{definition}\newtheorem{laws}{Law}[section]
  \theoremstyle{plain}\newtheorem{ju}[laws]{Jury}
  \theoremstyle{remark}\newtheorem*{marg}{Margaret}
\usepackage{esint} % 多重积分,需放在amsmath后

% 下划线宏包
\usepackage{ulem}
% LaTeX符号宏包
\usepackage{hologo}
	\newcommand{\xelatex}{\Hologo{XeLaTeX}}
	\newcommand{\bibtex}{\Hologo{BibTeX}}
% 其他符号
\usepackage{wasysym}
% 带箱小页
\usepackage{boxedminipage}
% 绘图
\usepackage{tikz}
	\usetikzlibrary{calc}
	\newcommand{\tikzline}[1]{{#1\tikz{\draw[#1,line width=9](0,0)--(0.5,0);}}, }

% 奇怪的小定义
\newcommand{\dpar}{\\ \mbox{}}	% 空两行
\newcommand{\qd}[1]{{\bfseries{#1}}}	% 强调
\newcommand{\co}[1]{{\bfseries{#1}}}   % Style of concept
\newcommand{\RED}[1]{{\color{cyan}{#1}}}
\newcommand{\cmmd}[1]{\fbox{\texttt{\char92{}#1}}}
\newcommand{\charef}[1]{第\ref{#1}章}
\newcommand{\secref}[1]{第\ref{#1}节}
\newcommand{\pref}[1]{第\pageref{#1}页}
\newcommand{\fref}[1]{图\ref{#1}}
\newcommand{\tref}[1]{表\ref{#1}}

% 编号列表宏包,并自定义了三个列表
\usepackage[inline]{enumitem}
	\setlist[enumerate]{label=\arabic* - ,font=\bfseries,itemsep=0pt}
	\setlist[itemize]{label=$\bullet$,font=\bfseries,leftmargin=\parindent}
	\setlist[description]{font=\bfseries\uline}

\newenvironment{fead}{\setlength{\parskip}{0pt}
	\begin{description}[font=\bfseries\uline,labelindent=\parindent]
		\setlength{\itemsep}{0pt}\setlength{\parsep}{0pt}\setlength{\parskip}{0pt}}
	{\end{description}}
% 带宽度的
\newenvironment{para}{\setlength{\parskip}{0pt}
	\begin{description}[font=\bfseries\ttfamily]
		\setlength{\itemsep}{0pt}\setlength{\parsep}{0pt}\setlength{\parskip}{0pt}}
	{\end{description}}
\newenvironment{feae}{\setlength{\parskip}{0pt}
	\begin{enumerate}[font=\bfseries,labelindent=0pt]}
	{\end{enumerate}}
\newenvironment{feai}{\setlength{\parskip}{0pt}
	\begin{itemize}[font=\bfseries]
		\setlength{\itemsep}{0pt}\setlength{\parsep}{0pt}\setlength{\parskip}{0pt}}
	{\end{itemize}}
\newenvironment{inlinee}
{\begin{enumerate*}[label=(\arabic*), font=\rmfamily, before=\unskip{:},itemjoin={{;}},itemjoin*={{,以及:}}]}
	{\end{enumerate*}。}

% 目录和章节样式
\usepackage{titlesec}
\usepackage{titletoc}   % 用于目录

\titlecontents{chapter}[1.5em]{}{\contentslabel{1.5em}}{\hspace*{-2em}}{\hfill\contentspage}
\titlecontents{section}[3.3em]{}{\contentslabel{1.8em}}
	{\hspace*{-2.3em}}{\titlerule*[8pt]{$\cdot$}\contentspage}
\titlecontents*{subsection}[2.5em]{\small}{\thecontentslabel{} }{}{, \thecontentspage}[;\qquad][.]
% 章节样式
\setcounter{secnumdepth}{3} % 一直到subsubsection
\newcommand{\chaformat}[1]{%
	\parbox[b]{.5\textwidth}{\hfill\bfseries #1}%
	\quad\rule[-12pt]{2pt}{70pt}\quad
	{\fontsize{60}{60}\selectfont\thechapter}}
\titleformat{\chapter}[block]{\hfill\LARGE\sffamily}{}{0pt}{\chaformat}[\vspace{2.5pc}\large
	\startcontents\printcontents{}{1}{\setcounter{tocdepth}{2}\songti}]
%\titleclass{\section}{top}
%\titleformat{\section}{\Large\bfseries}{\thesection}{0.5em}{}
\titleformat*{\section}{\centering\Large\bfseries}
\titleformat{\subsubsection}[hang]{\bfseries\large}{\rule{1.5ex}{1.5ex}}{0.5em}{}
% 扩展章节
\newcommand{\starsec}{\noindent\fbox{\S\textit{注意:本章节是一个扩展阅读章节。}}
	\\ \mbox{}}

\renewcommand{\contentsname}{目录}
	\renewcommand{\tablename}{表}
	\renewcommand\arraystretch{1.2}	% 表格行距
	\renewcommand{\figurename}{图}
% 设置不需要浮动体的表格和图像标题
\setlength{\abovecaptionskip}{5pt}
\setlength{\belowcaptionskip}{3pt}
\makeatletter
\newcommand\figcaption{\def\@captype{figure}\caption}
\newcommand\tabcaption{\def\@captype{table}\caption}
\makeatother
% 图表
\usepackage{array,multirow}
  \setlength\extrarowheight{2pt} % 行高增加
\usepackage{longtable}
\usepackage{graphicx}
  \graphicspath{{./tikz/}}
% 页面修正宏包
\usepackage{geometry}
\geometry{top = 1in}

% 代码环境
\usepackage{listings}
% Avoid copy line numbers of the listing code (Invalid for SumatraPDF Reader)
\usepackage{accsupp}
	\newcommand{\emptyaccsupp}[1]{\BeginAccSupp{ActualText={}}#1\EndAccSupp{}}
% Color
\usepackage{xcolor}
  \newcommand{\scol}[1]{\colorbox{#1}{\rule{0em}{1ex}}\,#1}
	\definecolor{commentcolor}{RGB}{85,139,78}
	\definecolor{numbercolor}{RGB}{166,206,168}
	\definecolor{stringcolor}{RGB}{206,145,108}
	\definecolor{keywordcolor}{RGB}{34,34,250}
	\definecolor{backcolor}{RGB}{220,220,220}
	\definecolor{packagecolor}{RGB}{0,128,0}
	\definecolor{envicolor}{RGB}{185,70,15}
% LaTeX Code Style
\lstset{language=[LaTeX]TeX,
		basicstyle=\small\ttfamily,
		commentstyle=\color{commentcolor},
		keywordstyle=\color{keywordcolor},
		stringstyle=\color{stringcolor},
		showstringspaces=false,
		% Package/Tikz-Lib Using
		classoffset=0,
		morekeywords={begin,end,usetikzlibrary},
		keywordstyle=\color{keywordcolor},
		classoffset=1,
		morekeywords={article,report,book,
			xeCJK,tikz,
			calc},
		keywordstyle=\color{packagecolor},
		classoffset=2,
		morekeywords={document,tikzpicture},
		keywordstyle=\color{envicolor},
		% Line Number Style
		numbers=left,
		numberstyle=\tiny\emptyaccsupp,
		stepnumber=1,
		% Frame and Background Color
		frame=single,
		framerule=0pt,
		backgroundcolor=\color{backcolor},
		% Spaces
		% belowskip=\medskipamount,
		emptylines=1,
		escapeinside=``}

\lstnewenvironment{latex}[1]{\lstset{#1}}{}
\newcommand{\latexline}[1]{{\lstinline[language=TeX,basicstyle=\small\ttfamily]{#1}}}

% Tikz Code
\lstdefinelanguage{tikzlang}{
	classoffset=0, % 蓝色的keyword
	morekeywords={begin,end,newcommand,
		draw,node,coordinate,tikzstyle,foreach},
	keywordstyle=\color{keywordcolor},
	classoffset=1, % 棕色的其他关键字
	morekeywords={tikzpicture,grid,at,
		thick,thin,very,ultra,
		red,green,yellow,blue,cyan,magenta,black,
		    gray,darkgray,lightgray,brown,lime,
		    olive,orange,pink,purple,teal,violet,white},
	keywordstyle=\color{envicolor},
	morecomment=[l]{\%},
	morecomment=[s]{/*}{*/},
	morestring=[b]',
	% Escape
	escapeinside=``
}
\lstnewenvironment{tikzcode}[1]{\lstset{language=tikzlang,basicstyle=\small\ttfamily,
		breaklines=true,%backgroundcolor=\color{white},
		linewidth=0.7\linewidth,#1}}{}

% 附录
\usepackage{appendix}
\renewcommand{\appendixname}{App.}

% 行号
\usepackage{lineno}

% 代码输入环境
\usepackage{verbatim,xcolor}
\newbox\savedlines
\newtoks\savedtokens
\makeatletter
\def\codeshow{%
\global\savedtokens={}%
\def\verbatim@processline{%
{\setbox0=\hbox{\the\verbatim@line}%
\hsize=\wd0
\the\verbatim@line\par}%
\global\savedtokens=\expandafter{\the\expandafter\savedtokens\the\verbatim@line^^J}}%
\@tempswatrue
\setbox0=\vbox\bgroup\parskip=0pt\topsep=0pt\partopsep=0pt
\verbatim}
\def\endcodeshow{\endverbatim%
\unskip\setbox0=\lastbox\egroup
\global\setbox\savedlines=\box0
\addvspace{1em}\par\noindent%
\colorbox{lightgray}{%
\begin{minipage}{.55\textwidth}{\usebox\savedlines}\end{minipage}}%
\hfill\fbox{\parbox{.40\textwidth}%
{\scantokens\expandafter{\the\savedtokens\unskip\endinput}}}%
\par\addvspace{1em}}
\makeatother

\usepackage{booktabs}

\bibliographystyle{plain}
\renewcommand{\bibname}{参考文献}
\usepackage[square,super,sort&compress]{natbib}

% 引用
\usepackage{hyperref}
\hypersetup{colorlinks, bookmarksopen = true, bookmarksnumbered = true, pdftitle=LaTeX-cn, pdfauthor=K.L Wu, pdfstartview=FitH}

\title{简单粗暴Excel}
\author{K.L Wu\\
  {\kaishu 本手册是\href{https://github.com/wklchris/Note-by-LaTeX}{wklchris-GitHub}的Excel-cn项目}
}
\date{最后更新于:\today}

\begin{document}
\maketitle

\setlength{\lineskiplimit}{0pt}
\tableofcontents
\setlength{\lineskiplimit}{3pt}

\chapter{Excel基础须知}
本手册基于Excel 2016,但对不早于2010的版本亦应有良好的兼容性。

\section{单元格引用}
四种单元格引用可以将光标放置在引用处,按F4键循环切换。
\begin{itemize}
\item 直接:A1
\item 固定列: \$A1
\item 固定行: A\$1
\item 固定单元格:\$A\$1
\end{itemize}

\vspace*{1ex}此外,用形如\texttt{A:A}的方式可以引用整个列。

\section{二元关系和通配符}
常用二元关系符有以下四种:\verb|<, =, >, <>|。\dpar

通配符有:
\begin{itemize}
\item 西文问号:任何一个字符。例如,sm?th 可找到``smith''和``smyth''。
\item 星号:任意的数量字符。例如,*east 可找到``Northeast''和``Southeast''。
\item 波浪号:转义字符。通过\verb|~?, ~*, ~~|可以找到问号、星号和波浪号。
\end{itemize}

\section{基本字符操作}
最简单的是“\&”符号,用于连接字符串。比如:\exstyle{A1\& A2, "a"\& "b"}。

\section{常用快捷操作}
包括鼠标和键盘的快捷操作。
\begin{description}
\item[F4] 1. 在输入状态下,可以在四种单元格引用方式之间切换;2. 在非输入状态,可以重复最近一次操作的内容。
\item[Shift + F8] 开启多选模式。无需按Ctrl,鼠标单击即可进行多选。
\item[Alt + =] 对当前单元格同列的上方所有单元格求和。
\item[Ctrl + ↑/↓] 跳到列首/列尾。可结合Shift使用。
\item[Ctrl + PgUp/PgDn] 切换工作表。
\item[Ctrl + Enter] 将公式应用到所选的所有单元格。
\item[Ctrl + Shift + Enter] 作为数组表达式执行。
\item[Ctrl + D] 将选中区域首个单元格的公式向下填充。
\end{description}

\vspace*{2ex}鼠标操作:
\begin{description}
\item[最适列宽:] 选中一列或多列,出现宽度改变指针后双击。
\item[多次格式刷:] 双击格式刷按钮,就能把格式复制给多个单元格。
\item[复制区域:] 选中,出现移动指针后,按住Ctrl,然后拖动。
\item[移动行/列:] 选中行/列,出现移动指针后,按住Shift,然后拖动。
\item[插入行/列:] 选中要在下方插入行的行(或要在右侧插入列的列),按住Shift,出现插入指针(相比宽度改变指针,在两箭头间多一个竖线)后,向下(或向左)拉动可以插入多行、列。
\item[删除行/列:] 类似,只是向上(或向左)拉动。
\end{description}

\chapter{数学计算}
\section{求和函数}
\subsection{求和:SUM}
函数\excel{SUM}用于求区域内数值的和。
\begin{excode}
= SUM(1, 2, 3)   
= SUM(A1, 3)
= SUM(A1:A2, B1:B2)
\end{excode}

跨多个工作表求和也是常见的用法:
\begin{excode}
= SUM(Sheet1:Sheet3!A1)  # 三个工作表的A1相加
= SUM('Sheet 1:Sheet 3'!A1)  # 当工作表名含空格时
\end{excode}

\subsection{单条件求和:SUMIF}
\label{fun:sumif}
函数\excel{SUMIF}将满足某个条件的单元格值相加。语法是:
\begin{syntax}
= SUMIF(range, criteria, [sum-range])
\end{syntax}

其中\exstyle{range}表示条件判断区域,\exstyle{criteria}表示条件;可选参数\exstyle{sum-range}表示实际求和区域。\RED{实际求和区域如果空缺,用条件判断区域代替}。

\begin{table}[!hbt]
    \centering
    \caption{SUMIF示例}\label{tab:sumif}
    \begin{tabular}{c|cc}
    \hline
      & A & B \\
    \hline
    1 & 1 & 2 \\
    2 & 3 & 4 \\
    3 & 5 & 6 \\
    \hline
    \end{tabular}
\end{table}

对于\autoref{tab:sumif}中的数据A1:B2,可以:
\begin{excode}
= SUMIF(A1:A2, ">1")  # A列中>1的数字之和。结果3
= SUMIF(A1:A2, ">1", B1:B2)  # 对应到B列,再求和。结果4
= SUMIF(A1:A3, ">1", B1:B2)  # 尺寸不同以A列为准。结果10
\end{excode}

\subsection{多条件求和:SUMIFS}
\label{fun:sumifs}
函数\excel{SUMIFS}可以将满足多个条件的单元格值相加。语法是:
\begin{syntax}
= SUMIFS(sum-range, cri-rg1, cri1, [cri-rg2, cri2], \ldots)
\end{syntax}

\begin{table}[!hbt]
    \centering
    \caption{SUMIFS示例}\label{tab:sumifs}
    \begin{tabular}{c|ccc}
    \hline
      & A & B & C \\
    \hline
    1 & 1 & 2 & 苹果\\
    2 & 3 & 4 & 苹果\\
    3 & 5 & 6 & 梨子\\
    \hline
    \end{tabular}
\end{table}

对于\autoref{tab:sumifs}的数据,有:
\begin{excode}
= SUMIFS(A1:A3, B1:B3, ">3")  # 因为4>3, 6>3,故结果3+5=8
= SUMIFS(A1:A3, B1:B3, ">3", C1:C3, "<>苹果")  # 结果5
\end{excode}

\subsection{乘积和:SUMPRODUCT}
函数\excel{SUMPRODUCT}将多个数组对应元素相乘,然后累加。

\begin{table}[!hbt]
    \centering
    \caption{SUMPRODUCT示例}\label{tab:sumproduct}
    \begin{tabular}{c|ccc}
    \hline
      & A & B & C \\
    \hline
    1 & 3 & 3 & 2\\
    2 & 2 & 2 & 3\\
    3 & 1 & 1 & 4\\
    \hline
    \end{tabular}
\end{table}

对于\autoref{tab:sumproduct}的数据,有:
\begin{excode}
= SUMPRODUCT(A1:A3, B1:B3)  # 结果`\greenmath{3\times 3+2\times 2+1\times 1=14}`
= SUMPRODUCT(A1:A3, B1:B3, C1:C3)  # 结果34
\end{excode}

\subsection{平方和:SUMSQ}
函数\excel{SUMSQ}用于求传入参数的平方和。比如:
\begin{excode}
= SUMSQ(3, 4)  # 结果`\greenmath{3^2+4^2=25}`
\end{excode}

\subsection{差加方和:SUMXMY2/SUMX2MY2/SUMX2PY2}
函数\excel{SUMXMY2}用于计算两数组对应元素之差的平方的和;函数\excel{SUMX2MY2}则用于计算两数组对应元素平方之差的和;函数\excel{SUMX2PY2}则用于计算两数组对应元素平方之和的和,相当于全体用\excel{SUMSQ}作平方和。

\begin{table}[!hbt]
    \centering
    \caption{SUMXMY2示例}\label{tab:sumxmy}
    \begin{tabular}{c|cc}
    \hline
      & A & B \\
    \hline
    1 & 1 & 2 \\
    2 & 3 & 4 \\
    3 & 5 & 6 \\
    \hline
    \end{tabular}
\end{table}

对于\autoref{tab:sumxmy}的数据,有:
\begin{excode}
= SUMXMY2(A1:A3, B1:B3)  # 结果`\greenmath{(1-2)^2+(3-4)^2+(5-6)^2=3}`
= SUMX2MY2(A1:A3, B1:B3)  # 结果`\greenmath{(1^2-2^2)+(3^2-4^2)+(5^2-6^2)=-21}`
= SUMX2PY2(A1:A3, B1:B3)  # 结果91。等同于SUMSQ(A1:B3)
\end{excode}

\section{求积和幂指函数}
\subsection{乘积:PRODUCT}
类似于函数\excel{SUM}的用法,只不过是乘积。例子:
\begin{excode}
= PRODUCT(A1:A2, B2, 0.5)
\end{excode}

\subsection{幂指函数:POWER/SQRT}
\begin{excode}
= POWER(8, 1/3)  # 结果`\greenmath{8^{\frac{1}{3}}=2}`
= SQRT(4)  # 结果`\greenmath{2}`
\end{excode}

\section{奇偶与求余函数}
\subsection{奇偶性判断:ISODD/ISEVEN}
返回值是TRUE或者FALSE。例如:\exstyle{ISODD(3)},结果是\exstyle{TRUE}。

\subsection{取余与求模:MOD/QUOTIENT}
函数\excel{MOD}的取余结果,符号与\qd{除数}相同。函数\excel{QUOTIENT}将商向零取整。
\begin{excode}
= MOD(5, -2)  # 结果`\greenmath{-1}` 
= QUOTIENT(5, 2)  # 结果`\greenmath{2}`
= QUOTIENT(-5, 2)  # 结果`\greenmath{-2}`
\end{excode}

\section{取整与舍入函数}
在本手册中,尽管并不规范,将使用\qd{向零/背零}表示绝对值减小/增大的取整方式,使用\qd{向下/向上}表示取整结果永远小于/大于传入值的取整方式。

\subsection{奇偶取整:ODD/EVEN}
函数\excel{ODD}背零舍入到最近的奇数;函数\excel{EVEN}背零舍入到最近的偶数。
\begin{excode}
= ODD(1.4)  # 结果`\greenmath{3}`
= EVEN(-2.4)  # 结果`\greenmath{-4}`
\end{excode}

\subsection{向下取整:INT}
函数\excel{INT}用于向下取整,取整结果总是不小于传入的数字。
\begin{excode}
= 3.56-INT(3.56)  # 返回正数的小数部分。结果`\greenmath{0.56}`
= INT(-2.1)  # 结果`\greenmath{-3}`
\end{excode}

\subsection{四舍五入:ROUND}
函数\excel{ROUND}用于四舍五入。
\begin{excode}
= ROUND(1.2345, 3)  # 结果`\greenmath{1.235}`
= ROUND(12.34, -1)  # 结果`\greenmath{10}`
\end{excode}

\subsection{截尾舍入:TRUNC}
函数\excel{TRUNC}截去指定数位之后的部分。数位参数为负,表示小数点之前的数位。
\begin{excode}
= TRUNC(1.2345,3)  # 结果`\greenmath{1.234}`
= =TRUNC(321.123,-1)  # 结果`\greenmath{320}`
\end{excode}

\subsection{向上/向下舍入:CEILING/FLOOR}
函数\excel{CEILING}用于向上取整;函数\excel{FLOOR}用于向下取整。语法:
\begin{syntax}
= CEILING(num, signif)
\end{syntax}

其中\exstyle{num}是传入值,舍入结果是\exstyle{signif}的整数倍。\excel{FLOOR}语法类似。

\begin{excode}
= CEILING(-2.6,0.5)  # 结果`\greenmath{-2.5}`
= CEILING(2.6,0.5)  # 结果`\greenmath{3}`
\end{excode}

\subsection{背零/向零舍入:ROUNDUP/ROUNDDOWN}
语法不同于\excel{CEILING}和\excel{FLOOR}:
\begin{syntax}
= ROUNDUP(num, digit)
\end{syntax}

其中\exstyle{num}是传入值,舍入到小数点后第\exstyle{digit}位。如果\exstyle{digit}是负数,那么舍入到小数点左侧相应位数。\excel{ROUNDDOWN}语法类似。

\begin{excode}
= ROUNDUP(-2.44,1)  # 背零,结果`\greenmath{-2.5}`
= ROUNDUP(2.44,1)  # 背零,结果`\greenmath{2.5}`
= ROUNDDOWN(-1.88,1)  # 向零,结果`\greenmath{-1.8}`
\end{excode}

\section{三角函数}
Excel 内置了圆周率\excel{PI}函数,参数为空,精度是15位小数。即:
\begin{syntax}
PI() = 3.14159265358979
\end{syntax}

\subsection{角度弧度转换:DEGREES/RADIANS}
函数\excel{DEGREES}将弧度转为角度,而\excel{RADIANS}则将角度转为弧度。
\begin{excode}
= DEGREES(PI())  # 结果`\greenmath{180}`
= RADIANS(90)  # 结果`\greenmath{\frac{\pi}{2}=1.5707963267949}`
\end{excode}

\subsection{正弦/余弦/正切/余切/正割/余割:SIN/COS/TAN/COT/SEC/CSC}
\begin{excode}
= SIN(PI()/2)  # 结果`\greenmath{1}`
= DEGREES(ASIN(1))  # 结果`\greenmath{0}`
= TAN(RADIANS(45))  # 结果`\greenmath{1}`
= COT(RADIANS(45))  # 结果`\greenmath{1}`
\end{excode}

\subsection{反三角函数:ASIN/ACOS/ATAN/ACOT}
拒绝赘述。

\section{随机数:RAND/RANDBETWEEN}
函数\excel{RAND}用于生成\RED{大于等于0、小于1}的实数;函数\excel{RANDBETWEEN}用于生成\RED{介于两参数之间(闭区间)}的整数。
\begin{excode}
= RAND()*(5-1)+1  # 生成大于等于1、小于5的随机实数
= RANDBETWEEN(1, 10)  # 生成大于等于1、小于10的随机整数
\end{excode}

\section{统计函数}
前面介绍的\excel{SUM}也可以算作统计函数。

\subsection{均值:AVERAGE/AVERAGEA}
若现有$m+n$条数据,其中$m$条是数值,$n$条不是数值,那么均值有:
\begin{itemize}
\item 数值均值:函数\excel{AVERAGE}将忽略$n$条数据,只计算数值数据。
\item 全体均值:函数\excel{AVERAGEA}计算均值时,分母仍为$m+n$。
\end{itemize}

\begin{table}[!hbt]
    \centering
    \caption{AVERAGE示例}\label{tab:average}
    \begin{tabular}{c|ccccc}
    \hline
      & A & B & C & D & E\\
    \hline
    1 & 1 & 2 & 3 & 4 & N/A\\
    \hline
    \end{tabular}
\end{table}

对于\autoref{tab:average}的数据,有:
\begin{excode}
= AVERAGE(A1:E1)  # 结果`\greenmath{\frac{1+2+3+4}{4}=2.5}`
= AVERAGEA(A1:E1)  # 结果`\greenmath{\frac{1+2+3+4}{5}=2}`
\end{excode}

\subsection{条件均值:AVERAGEIF/AVERAGEIFS}
语法与\autoref{fun:sumif}以及\autoref{fun:sumifs}相似,不再赘述。

\subsection{计数函数:COUNT/COUNTA/COUNTBLANK}
函数\excel{COUNT}统计\RED{数字单元格}的数量;函数\excel{COUNTA}统计\RED{非空单元格}的数量\footnote{注意:即使单元格内的公式结果是空字符,该单元格也\qd{不是}空的。};而函数\excel{COUNTBLANK}统计\RED{空单元格}的数量。

\subsection{条件计数:COUNTIF/COUNTIFS}
语法与\autoref{fun:sumif}以及\autoref{fun:sumifs}相似,不再赘述。

\subsection{频数统计:FREQUENCY}
函数\excel{FREQUENCY}是一个数组函数,用于统计落在多个区间内的数据量。参考

\subsection{极值与分位值:MAX/MIN/LARGE/SMALL}
函数\excel{MAX}和\excel{MIN}求区域最大和最小值。函数\excel{LARGE}和\excel{SMALL}则用于求区域第$k$大值或第$k$小值。

\begin{excode}
= SMALL({1,2,3,4},3)  # 第三小的数。结果`\greenmath{3}`
\end{excode}

\subsection{条件极值:DMIN/DMAX}
语法:
\begin{syntax}
= DMIN(database, field, criteria)
\end{syntax}

其中,\exstyle{database}是数据总表,\exstyle{field}是要寻找极值的列名,\exstyle{criteria}是条件约束区。函数\excel{DMAX}语法类似。

\begin{table}[!hbt]
    \centering
    \caption{DMIN示例}\label{tab:dmin}
    \begin{tabular}{c|cccc}
    \hline
      & A & B & C & D \\
    \hline
    1 & 树种 & 高度 & 产量 & 高度 \\
    2 & 苹果树 & $>10$ & & $<16$ \\
    3 & 梨树 & \\
    4 & \\
    5 & 树种 & 高度 & 产量 &  \\
    6 & 苹果树 & 18 & 14 \\
    7 & 梨树 & 12 & 10 \\
    8 & 樱桃树 & 13 & 9 \\
    9 & 苹果树 & 14 & 10 \\
    10 & 梨树 & 9 & 8 \\
    11 & 苹果树 & 8 & 6 \\
    \hline
    \end{tabular}
\end{table}

对于\autoref{tab:dmin}的数据,有:
\begin{excode}
# 寻找高度在10与16间的苹果树、或者任何梨树,之中的最小产量
= DMIN(A5:C11, "产量", A1:D3)  
\end{excode}

\subsection{中位数与众数:MEDIAN/MODE}
函数\excel{MEDIAN}返回中位数,函数\excel{MODE}返回众数。注意,\excel{MODE}函数在有多个众数时也只返回一个值(函数\excel{MODE.MULT}可以解决这个问题,参考\autoref{fun:mode.mult})。

\subsection{排名:RANK.AVG/RANK.EQ}
旧版本的函数\excel{RANK}在将来可能不再受支持。新版本的\excel{RANK.AVG}返回平均排名;函数\excel{RANK.EQ}则返回同一排名(可能造成某些排名空缺)。

语法:
\begin{syntax}
= RANK.AVG(num, ref, [order])
\end{syntax}
其中,\exstyle{num}是要进行排名的数值,\exstyle{ref}是数据列表,可选参数\exstyle{order}非零时返回升序下的排名,为零或省略时返回降序下的排名。

\begin{table}[!hbt]
    \centering
    \caption{RANK示例}\label{tab:rank}
    \begin{tabular}{c|ccccc}
    \hline
      & A & B & C & D & E\\
    \hline
    1 & 1 & 2 & 2 & 3 & 4\\
    \hline
    \end{tabular}
\end{table}

对于\autoref{tab:rank}的数据,有:
\begin{excode}
= RANK.AVG(2,A1:E1)  # 降序,2在第3、4位。结果3.5(平均排名)
= RANK.EQ(2,A1:E1)  # 降序,2在第3、4位。结果3(同一排名)
= RANK.EQ(2,A1:E1,1)  # 升序,2在第2、3位。结果2(同一排名)
\end{excode}

\section{其他函数}
\subsection{绝对值:ABS}
拒绝赘述。

\subsection{指数函数:EXP}
自然对数的底通过\exstyle{EXP(1)}来表示,其他拒绝赘述。

\subsection{排列组合:COMBIN/PERMUT}
\begin{excode}
= COMBIN(4, 2)  # 结果`\greenmath{\mathrm{C}_4^2 = \frac{4\times 3}{2} = 6}`
= PERMUT(4, 2)  # 结果`\greenmath{\mathrm{P}_4^2 = 4\times 3 = 12}`
\end{excode}

\subsection{进制转换:BASE}
函数BASE用于转换进制,语法:
\begin{syntax}
= BASE(number, radix, [min-length])
\end{syntax}

其中\exstyle{number}是将进行转换的数字;\exstyle{radix}是基数,必须是整数且$radix\in [2, 36]$;可选参数\exstyle{min-length}用于指定返回值的最小位数。

\begin{excode}
= BASE(10, 2)  # 转二进制,结果`\greenmath{1010}`
= BASE(10, 8, 4)  # 转八进制并补齐四位,结果`\greenmath{0012}`
\end{excode}

\section{数组函数}
数组函数在确认公式时,需要借助Ctrl + Shift + Enter(下文称CSE)而不是普通的Enter。数组的表示方法很简单:一对外侧的花括号,内部用逗号表示列分隔,用分号表示行分隔。通常地,在Excel中选中的任一单元格区域都能看做一个数组。

\subsection{矩阵点乘:MMULT}
\begin{syntax}
= MMULT(array1, array2)
\end{syntax}

其中,array1区域的行数必须等于array2区域的列数;这也是矩阵点乘的定义所要求的。你需要确认计算结果是几行几列的,然后选择同等大小的区域,在编辑模式下按CSE进行应用。

\begin{table}[!hbt]
    \centering
    \caption{MMULT示例}\label{tab:mmult}
    \begin{tabular}{c|cccc}
    \hline
      & A & B & C & D \\
    \hline
    1 & 1 & 2 & 1 & 2 \\
    2 & 3 & 4 & 3 & 4 \\
    \hline
    \end{tabular}
\end{table}

对于\autoref{tab:mmult}的数据,选中\exstyle{F1:G2},再输入:
\begin{excode}
= MMULT(A1:B2,C1:D2)  # 按CSE。结果是:`\greenmath{\{7, 10; 15, 22\}}`
\end{excode}

\subsection{众数数组:MODE.MULT}
\label{fun:mode.mult}
选中一个竖直的区域作为返回值区域,然后用CSE组合键就能显示所有众数。例如,选中A1:B1,输入以下公式再按CSE:
\begin{excode}
= MODE.MULT({1,2,2,3,3,4,6})  # 结果`\greenmath{\{2; 3\}}`
\end{excode}

\subsection{频数数组:FREQUENCY}
\label{fun:frequency}
选中一个竖直的区域作为返回值区域,然后用CSE组合键就能显示各区间的频数。例如,选中A1:C1,输入以下公式再按CSE:
\begin{excode}
= FREQUENCY({1,2,3,8,10,15},{5,10})  # 结果`\greenmath{\{3; 2; 1\}}`
\end{excode}

以上公式的含义:在数据集$1,2,3,4,10,15$中,小于等于5的数有3个,大于5小于等于10的数有2个,大于10的数有一个。

\end{document}